\section{Methods}

\subsection{Bacterial growth}

All \textit{Escherichia coli} (\textit{E.coli.}) strains were grown in LB media at 37$^\circ$C. Bacteria transformed with plasmids containing ampicillin or kanamycin resistant genes were selected on growth media containing 0.2 mg/mL ampicillin or 0.05 mg/mL kanamycin, respectively.

\subsection{Preparation of competent bacterial cells}

A single bacterial colony was inoculated into 5 mL LB and incubated overnight at 37$^\circ$C and 250 rpm. The next day, 100 $\mu$L overnight culture were inoculated into 5 mL fresh LB and incubated at 37$^\circ$C and 250 rpm until the OD\sub{600} reached 0.3. Then 1 mL culture was centrifuged on a table-top micro-centrifuge at 12,000 rpm, 1 minute. The supernatant was discarded, and 100 $\mu$L 0.1 M CaCl\sub{2} were added to re-suspend the bacteria cells. Cells were put on ice for at least 2 hours before use.

\subsection{Transformation of competent bacterial cells}

10 ng plasmid DNA or 5 $\mu$L of a ligation reaction, were added to 100 $\mu$L DH5$\alpha$ (F-$\Phi80lac$Z$\Delta$M15) $\Delta$(\textit{lac}ZYA-\textit{arg}F) U169 \textit{rec}A1 \textit{end}A1 \textit{hsd}R17 (rK-, mK+) \textit{pho}A \textit{sup}E44 $\lambda$-\textit{thi}-1 \textit{gyr}A96 \textit{rel}A1) competent cells. After incubation on ice for 30 minutes, cells were exposed to heat shock at 42$^\circ$C for 50 seconds. Cells were immediately placed back on ice for 2 minutes and 100 $\mu$L of LB medium were added. Cells were incubated for 1 hour at 37 $^\circ$C and pelleted at 12,000 rpm for 60 seconds, re-suspended in 50 $\mu$L of LB and spread onto agar plates containing the appropriate antibiotics and grown overnight at 37$^\circ$C.

\subsection{Isolation of plasmid DNA, restriction digestion, gel extraction, and DNA ligations}

Plasmid DNA was isolated from transformed DH5$\alpha$ cells using Qiagen miniprep and Promega midiprep kits according to the manufacturers' instructions. 1 $\mu$g of DNA was cleaved using 5 U of the restriction endonucleases and appropriate buffer in accordance with the manufacturers' recommendations (Fermentas, and NEB). Digestions were adjusted to a final volume of 15 $\mu$L using dH\sub{2}O and incubated at 37$^\circ$C for at least 1.5 hours. The products were resolved on 1\% standard agarose (Invitrogen) gels for examining DNA fragments. Gels were made with 1$\times$ TAE buffer and run in the same buffer for 50 minutes at 70V, stained with 0.5 $\mu$g/mL ethidium bromide for 10 minutes, and visualized under UV light. For purification of DNA fragments after gel electrophoresis, the DNA fragments were isolated from the gel using the Qiaquick gel extraction kit (Qiagen) following the manufacturer's protocol. DNA ligation reactions were performed at room temperature using the T4-DNA ligase system (NEB). Ligation reactions contained insert and vector DNA at a molar ratio between 3:1 to 5:1, 2.5 U T4-DNA ligase and appropriate buffer in a total volume of 10 $\mu$L.

\subsection{Protein expression and purification}

BL21-CodonPlus-RIL bacteria (\textit{E.Coli.} B F- \textit{omp}T \textit{hsd}S(r\sub{B-} m\sub{B-}) \textit{dcm}+ \textit{Tet}\sus{r} \textit{gal} $\lambda$ \textit{end}A \textit{Hte} [\textit{arg}U \textit{ile}Y \textit{leu}W \textit{Cam}\sus{r}]) were transformed with appropriate plasmids. A single colony was inoculated into 5 mL LB containing appropriate antibiotics and 34 $\mu$g/mL chloramphenicol, and incubated in 37$^\circ$C at 250 rpm overnight. Then 500 $\mu$L overnight culture were inoculated into 50 mL fresh LB containing appropriate antibiotics and 34 $\mu$g/mL chloramphenicol. The bacteria were cultured at 37$^\circ$C at 250 rpm. When the OD\sub{600} of the culture reached 0.5 - 0.7, 0.5 mM IPTG was added. The protein was induced at 25$^\circ$C for 4 hours. Bacterial cells were pelleted at 4000 rpm, 4$^\circ$C, for 15 minutes.

For the purification of 6$\times$His tagged protein, pelleted bacterial cells were lysed in 2 mL 1$\times$ Ni Binding Buffer supplied with 0.1\% NP-40 and protease inhibitors cocktail. Cells were disrupted by sonication using Misonix XL-2000 (Microson\sus{TM}) on ice until the lysate became clear. The lysate was centrifuged at 13,200 rpm, 4$^\circ$C for 15 minutes. The debris was removed and the supernatant was incubated with 500 $\mu$L Ni-Agarose (Qiagen) on a rotating platform at 4$^\circ$C for 1 hour, and then the beads were washed with 1$\times$ Ni Wash Buffer for 6 times. 50 $\mu$L 1$\times$ Ni Elution Buffer were added to the beads, and the elution was done at 4$^\circ$C on a rotating platform for 5 minutes (Elute 1). Then the elution was repeated in 100 $\mu$L 1$\times$ Ni Elution buffer for 10 minutes (Elute 2). Then the elution was repeated for a third and fourth time in in 100 $\mu$L 1$\times$ Ni Elution buffer for 10 minutes respectively (Elutes 3 and 4). All four elutes were kept separately and dialysed against 1$\times$ PBS overnight. After the dialysis, glycerol was added to a final concentration of 30\% (v/v), and proteins were stored in -80$^\circ$C. Only Elutes 2, 3 and 3 were used in the downstream experiments.

For the purification of GST tagged protein, pelleted bacterial cells were lysed in 2 mL 1$\times$ PBS supplied with 0.5\% Triton and protease inhibitor cocktail. Cells were disrupted by sonication using Misonix XL-2000 (Microson\sus{TM}) on ice until the lystate became clear. The lysate was centrifuged at 13,200 rpm, 4$^\circ$C for 15 minutes. The debris was removed and the supernatant was incubated with 300 $\mu$L Glutathione-Agarose (Sigma) on a rotating platform at 4$^\circ$C for 1 hour, and then the beads were sequentially washed once with 1$\times$ PBS/0.5\% Triton, four times with 1$\times$ PBS/400 mM NaCl, once with Pulldown Buffer. The protein- bound beads were stored in the Pulldown Buffer at 4$^\circ$C for the future use.

\subsection{GST pulldown assay}

U2OS cells were lysed in the Pulldown Buffer by sonication using Bioruptor UCD200 (Diagenode) at high power, 30 seconds ON/OFF, 5 minutes. One 10-cm dish of U2OS cells (lysed in 500 $\mu$L Pulldown Buffer) were used per pulldown. After removing the cell debris by centrifugation, the cell lysate was incubated with \textasciitilde 1 $\mu$g GST or GST fusion proteins on glutathione-agarose beads (30 $\mu$L) at room temperature for 2 hours. After washing 5 times with the Pulldown Buffer, 20 $\mu$L 2$\times$ SDS loading buffer was added to the beads, and proteins were eluted by boiling the beads at 99$^\circ$C for 10 minutes. Then proteins were analysed by SDS-PAGE and western blot.

\subsection{Electrophoretic Mobility Shift Assay (EMSA)}

\textbf{(1) End-labelling of DNA binding sites:} 1 $\mu$g of each short complementary oligo was mixed with H\sub{2}O in a total volume of 20 $\mu$L. Then the oligo mixture was denatured in a 100$^\circ$C water bath for 3 minutes and annealed by cooling slowly overnight. 10 $\mu$L labelling reaction was set up as:

\begin{tabular}{>{\raggedleft\arraybackslash}m{1cm}>{\raggedright\arraybackslash}m{1.5cm}>{\raggedright\arraybackslash}m{8cm}}
    20 & $\mu$Ci      & \sus{32}P-dCTP\\
    1 & U             & Klenow\\
    1 & $\times$       & Klenow Buffer\\
    0.2 & $\mu$g   & Annealed oligo\\
    1 & $\mu$L        & 2 mM dNTP(-dCTP)\\
\end{tabular}

The reaction was incubated at 37$^\circ$C for 30 minute.

\textbf{(2) Purification of labelled probes:} After end-labelling reaction, 4 $\mu$L Bandshift Load
Buffer were added, and the probes were run on a 10\% polyacrylamide gel using 1$\times$ TBE and exposed to X-ray film (Kodak). The parts of the gel containing the labelled DNA duplexes were excised, transferred into 1.5 ml Eppendorf tubes and cut into small pieces. 400 $\mu$L 1$\times$ TE were added to elute the probes from the gel. The elution was performed on a shaker at 1400 rpm at room temperature, overnight. After elution, the samples were spun through glass wool to remove the gel debris. The elutes which contained labelled DNA duplexes were ethanol precipitated using 1/10 volume of 3 M potassium acetate and 2.5 volumes of absolute ethanol. After washing with 70\% ethanol, the labelled DNA duplexes were resuspended in 200 $\mu$L 1$\times$ FpF buffer. 1 $\mu$L of each probe was loaded onto a filter paper and exposed to phosphorimager screen for 10 minutes, and the signals were quantified by Quantity One (Bio-Rad) software to compare labelling efficiencies among different probes.

\textbf{(3) EMSA reaction:} 10 $\mu$L protein mix was set up first:

\begin{tabular}{>{\raggedleft\arraybackslash}m{1cm}>{\raggedright\arraybackslash}m{1.5cm}>{\raggedright\arraybackslash}m{8cm}}
    3.2 & $\mu$L   & Dz Buffer\\
    2.5 & $\mu$L   & 4$\times$ FpF Buffer\\
    1   & $\mu$L   & Poly dI/dC (250 $\mu$g/mL)\\
    0.8 & $\mu$L   & BSA (20 mg/mL)\\
    0.5 & $\mu$L   & 1.5 M KCl (final concentration 75 mM)\\
    2   & $\mu$L   & Appropriate amount of purified protein\\
\end{tabular}

The reaction was incubated on ice for 10 minutes. Then 2 $\mu$L \sus{32}P-labelled DNA duplexes were added to the reaction, and the reaction was incubated at room temperature for 20 minutes and an additional 20 minutes when 1 $\mu$L anti-FLAG M2 antibody was added. For competition assays, 2 $\mu$L labelled DNA duplexes and 2 $\mu$L varying concentrations of unlabelled competitors were added to the 10 $\mu$L protein mix, and the reaction was incubated at room temperature for 40 minutes. 2 $\mu$L Bandshift Loading Buffer were added, and the samples were loaded on to a continuously running 5\% polyacrylamide gel (pre-run at least 30 minutes) in 0.25$\times$ TBE running buffer in the cold room, at 250 V for 3 hours. The gels were fixed in FIX Solution at room temperature for 20 minutes and dried on a filter paper. The gels were exposed overnight to a phosphorimager screen and the signals were quantified by Quantity One (Bio-Rad) software.

\subsection{Polymerase Chain Reaction (PCR) for cloning} \label{section:pcrclone}

Standard PCR reactions were performed. Each PCR was carried out in a 50 $\mu$L reaction containing:

\begin{tabular}{>{\raggedleft\arraybackslash}m{1cm}>{\raggedright\arraybackslash}m{1.5cm}>{\raggedright\arraybackslash}m{8cm}}
    1  & $\mu$L    & 10 mM dNTP\\
    5  & $\mu$L    & 10$\times$ Pfu Ultra Buffer\\
    1  & $\mu$L    & Pfu Ultra II polymerase (2.5 U/$\mu$L)\\
    2  & $\mu$L    & 5 $\mu$M primer mix\\
    1  & $\mu$L    & DNA template (50 ng appropriate plasmid)\\
    40 & $\mu$L    & H\sub{2}O\\
\end{tabular}

The reaction was incubated at the following temperature profile: 99$^\circ$C 3 minutes, [99$^\circ$C 30 seconds, 56$^\circ$C 30 seconds, 72$^\circ$C 3 minutes] $\times$ 28, 72$^\circ$C 5 minutes.

\subsection{Cell culture and transfection}

Human osteosarcoma U2OS cells and embryonic kidney HEK293/293T cells, and cervical cancer HeLa cells were cultured in Dulbecco’s modified Eagle’s medium (DMEM) (Gibco) supplemented with 10\% fetal calf serum (FCS) (Gibco).

\textbf{(1) Plasmid transfection:} For transient transfection, cells were plated in either six-well plates or 10-cm dishes, and transfected with Polyfect (Qiagen) transfection reagent (for HEK293/293T cells) or X-tremeGene HP (Roche) transfection reagent (for U2OS cells) according to the manufacturer's protocol. Cells were harvested 24 hours or 48 hours after transfection for subsequent analyses. For stable transfection, cells were plated in six-well plates. 48 hours after transient transfection, cells were trypsinised, and split at 1:50 ratio into media containing appropriate selection antibiotics (see \textbf{Section \ref{section:antibiotics}, Table \ref{table:tc}}). The media were changed every 3 days, until colonies of cells were formed. Single colonies were picked, or a pool of colonies (if using the Flip-In\sus{TM} system) was combined. Stable cell lines were maintained in the media containing the selection antibiotics.

\textbf{siRNA transfection:} All siRNAs were used at 20 nM final concentration. Cells were transfected with siRNAs using Lipofectamine RNAiMAX (Invitrogen) using the reverse-transfection method according to the manufacturer's instructions. 48 hours to 72 hours after transfection, cells were harvested for subsequent analyses.

\subsection{Luciferase assay}

$10^5$ U2OS cells were seeded per well in a 12-well plate. Each well of cells were transfected with 200 ng luciferase reporters (pGL4.1), 10 ng pRL-Renilla (Promega), and 790 ng plasmids encoding the appropriate proteins using X-tremeGene HP (Roche). 24 hours after the transfection, cell extracts were collected using the Dual-Luciferase Reporter Assay kit (Promega, E1910), and experiments were carried on according to the manufacturer's instructions (Promega, TM040). Due to the deregulation of the Renilla reporter after the overexpression of FOXM1, the luciferase activities were normalised by the protein concentrations of the extracts. The protein concentration was measured by the BCA Protein Assay Reagent kit (Thermo Scientific \#23225) according to the manufacturer’s instructions. All samples were done in triplicates and all experiments were independently repeated at least twice.

\subsection{Immunofluorescence}

U2OS cells were grown on the glass cover slips. Cells were fixed with 4\% paraformaldehyde at the room temperature for 15 minutes, and permeabilised by 1×PBS containing 0.2\% (v/v) Triton at room temperature for 15 minutes. After blocking the cells with 1$\times$ PBS containing 5\% (w/v) BSA, cells were incubated with the appropriate primary antibodies at room temperature for 1 hour. Then cells were washed with 1$\times$ PBS for three times and 1$\times$ PBS containing 0.1\% (v/v) TWEEN-20 once. Cells were incubated with the appropriate fluoro-tagged secondary antibodies at room temperature for 30 minutes, and washed three times with 1$\times$ PBS and once with 1$\times$ PBS containing 0.1\% (v/v) TWEEN-20. The cover slip was mounted on a microscope slide using ProLong\textregistered Gold Antifade Reagent with DAPI (Invitrogen). The following primary antibodies were used at a 1 in 400 dilution: FLAG M2 (Sigma) and FOXO3 (75D8) (Cell Signalling). The following secondary antibodies were used at a 1 in 400 dilution: Alexa Fluor 594 donkey anti-mouse (Invitrogen) and Alexa Fluor 488 donkey anti-rabbit (Invitrogen). Images were acquired on a Delta Vision $\times$3 (Applied Precision) restoration microscope. The images were collected using a Coolsnap HQ (Photometrics) camera with a Z optical spacing of 0.2 $\mu$m. Raw images were then deconvolved using the Softworx software.

\subsection{Cell cycle arrest, synchronisation, and FACS analysis}

For the G0/G1 phase arrest, cells were maintained in DMEM without any FCS for 48 hours; for the S phase arrest, cells were maintained in complete media containing 5 mM Hydroxyurea (Sigma-Aldrich) for 48 hours; for the G2/M phase arrest, cells were maintained in complete media containing 100 ng/mL nocodazole (Sigma-Aldrich) for 16 hours, followed by mitotic shake-off (\cite{zwanenburg1983standardized}), during which the mitotic cells would detach from the surface of the culture dish, while the G2 phase cells would stay attached; G1 phase synchronisation was performed using a double thymidine block by culturing 30\% confluent cells for 12 - 16 hours in complete media containing 2 mM thymidine (Sigma-Aldrich), 9 - 12 hours in media lacking thymidine, followed by an additional 10 - 12 hours in the presence of 2 mM thymidine. For analysis by flow cytometry, collected cells were fixed in 70\% ethanol overnight. For DNA-staining, cells were treated with 50 $\mu$g RNase A and 50 $\mu$g propidium iodide in a total volume of 400 $\mu$L PBS and incubated at 37$^\circ$C for 30 minutes. Samples were examined on a CYAN-Calibur flow cytometer and data were analysed using ModFit software (ModFit LT).

\subsection{SDS-PAGE electrophoresis and western blot analysis}

Total cell lysates were prepared by harvesting cells into Triton Lysis Buffer containing phosphatase inhibitors and protease inhibitor cocktail (Roche). For $\lambda$ phosphatase treatment, total cell lysates were prepared by harvesting cells into Triton Lysis Buffer with or without phosphatase inhibitors, and the cell lysates were treated with 200 U $\lambda$ phosphatase (NEB) in a 20 $\mu$L reaction at $^\circ$C for 2 hours according to the manufacturer's instructions. Protein concentration was measured in a BCA assay (Thermo Scientific) according to the manufacturer's protocol, and a total of 15 - 30 $\mu$g protein diluted in 1$\times$ SDS-PAGE loading buffer were resolved on 8-12\% SDS-PAGE gels. Proteins were transferred to a nitrocellulose membrane, which was later probed with appropriate antibodies. Signals were detected by the Odyssey Imaging System (Licor Biosciences).

\subsection{Co-immunoprecipitation (CoIP)}

Cells were cultured in 10-cm dishes, and plasmids encoding FLAG-tagged or GFP-tagged proteins were transfected. 24 hours after the transfection, cells were lysed with 400 $\mu$L CoIP Buffer and sonicated in the Bioruptor UCD200 (Diagenode) for 5 minutes, high power. The cell debris was removed by centrifugation at 13,200 rpm, 4$^\circ$C, for 10 minutes. For the co-immunoprecipitation with GFP-tagged proteins, the Miltenyi GFP Kit (Miltenyi Biotec) was used according to the manufacturer’s instructions. For the co- immunoprecipitation with FLAG-tagged proteins, 30 $\mu$L FLAG M2-Agarose (Sigma) were added to each IP and incubated at 4$^\circ$C on a rotating platform overnight. Each IP was washed five times with CoIP buffer and then boiled in 15 $\mu$L 2$\times$ SDS-PAGE loading buffer. Samples were analysed by western blot.

\subsection{Chromatin immunoprecipitation (ChIP)} \label{chipmethod}

Generally, $2 \textmd{ to } 5 \times 10^6$ cells were used per ChIP. 0.5 $\mu$g antibody and 10 $\mu$L Dynabeads (Invitrogen) were used to immunoprecipitate the chromatin prepared from $5 \times 10^6$ cells. Experiments were performed according to the published protocol with some modification (\cite{lee2006chromatin}):

\textbf{(1) Preparation of antibody-beads complex:} 0.5 $\mu$g antibody was incubated with 10 $\mu$L Pan Mouse IgG Dynabeads (for mouse IgG1 monoclonal antibody), Protein A Dynabeads (for rabbit polyclonal antibody), or Protein G Dynabeads (for other antibodies) in ChIP Blocking Buffer overnight (at least 6 hours). The antibody-beads complex was washed with ChIP Blocking Buffer for three times and balanced with ChIP Lysis Buffer III supplied with 1\% Triton (v/v) in 4$^\circ$C until use.

\textbf{(2) Crosslinking of cells:} For U2OS cells, culture media were removed and Crosslinking Solution was added to the culture dish; for HEK293/293T cells, stock Formaldehyde solution (Sigma) was added directly to the culture media at a final concentration of 1\%. Crosslinking was performed at room temperature for 10 minutes with slow rocking. Then crosslinking was quenched by 0.125 M Glycine at room temperature for 5 minutes with slow rocking.

\textbf{(3) Chromatin preparation and sonication:} Cells were scraped into 6 mL ice cold PBS with Protease Inhibitor Cocktail, and centrifuged at 800 g for 10 minutes at 4$^\circ$C. Cells were washed with ChIP Lysis Buffer I, ChIP Lysis Buffer II, and lysed in ChIP Lysis Buffer III ($5 \times 10^6$ cells in 500 $\mu$L ChIP Lysis Buffer III). Chromatin was sonicated using Bioruptor UCD200 (Diagenode) at high power, 30 seconds ON/OFF, 15 minutes. Then 1/10 volume of 10\% Triton X-100 was added (\textit{e.g.} 50 $\mu$L 10\% Triton X-100 was added to 500 $\mu$L sonicated chromatin), and centrifuged at 20,000g for 10 minutes, 4$^\circ$C. Debris was removed.

\textbf{(4) Chromatin immunoprecipitation:} 50 $\mu$L lysate were saved as Input, the remaining 500 $\mu$L lysate were added to the antibody-beads complex, and incubated on a rotator at 4$^\circ$C overnight.

\textbf{(5) Wash, elution, and crosslink reversal:} Each IP was washed five times with 500 $\mu$L ChIP RIPA Wash Buffer, once with 1$\times$ TE + 50 mM NaCl. All wash steps were performed at 4$^\circ$C, and each wash was done by inverting the tubes 15 - 20 times. Then 100 $\mu$L ChIP Elution Buffer were added to the tube, and the elution was performed on a Eppendorf Thermomixer (Eppendorf) at 65$^\circ$C, 1400 rpm shaking, 25 minutes. Tubes were centrifuged at 16,000g for 1 minute, and elutes were transferred to new tubes. 100 $\mu$L ChIP Elution Buffer were added to the beads, and elution was repeated once again. Two elutes from the same IP were pooled and put into 65$^\circ$C oven overnight to reverse crosslink. For Input sample, 3 volumes of ChIP Elution Buffer were added to the sample, and crosslink reversal was performed in a 65$^\circ$C oven overnight.

\textbf{(6) Proteinase K digestion and DNA purification:} Proteinase K (Roche) was added to a final concentration of 0.2 $\mu$g/mL. The digestion was done at 45$^\circ$C for 1 hour. After proteinase K treatment, 1$\times$ TE was added to top up the volume to 400 $\mu$L. Then 400 $\mu$L Phenol:Chloroform:isoamyl alcohol mixture (Sigma 77617) were added, and vigorously vortexed until an emulsion formed. The mixture was centrifuged at 20,000g for 10 minutes at room temperature. The aqueous phase was transferred to a new tube containing 16 $\mu$L 5 M NaCl, 20 $\mu$g glycogen, and 800 $\mu$L ethanol. The mixture was incubated at -80$^\circ$C for 1 hour, and centrifuged at 16,100 g at 4$^\circ$C for 30 minutes. The DNA pellet was washed with 500 $\mu$L 70\% ethanol and centrifuged at 16,100 g for 10 minutes. The ethanol was discarded, and the pellet was dried on the bench and resuspended in 50 $\mu$L 10 mM Tris, pH 8.5.

\textbf{(7) Quantitative PCR (qPCR):} 2 $\mu$L of DNA from ChIP samples were used for each PCR reaction. The concentration of Input sample was measured by NanoDrop Spectrophotometer ND-1000 (Thermo Scientific). 50 ng of Input DNA was serially diluted to make the standard curve for each primer. The reaction contained:

\begin{tabular}{>{\raggedleft\arraybackslash}m{1cm}>{\raggedright\arraybackslash}m{1.5cm}>{\raggedright\arraybackslash}m{8cm}}
    5    & $\mu$L    & SYBR PCR Master Mix (Qiagen)\\
    0.2  & $\mu$L    & 50 mM MgCl\sub{2} (Bioline)\\
    0.6  & $\mu$L    & 50 $\mu$M primer mix\\
    2.2  & $\mu$L    & H\sub{2}O\\
    2    & $\mu$L    & DNA template\\
\end{tabular}

The reaction was incubated at the following temperature profile: 95$^\circ$C 15 minutes, [95$^\circ$C 10 seconds; 53$^\circ$C 15 seconds; 72$^\circ$C 20 seconds] $\times$ 40 cycles; melting curve: 72 - 99$^\circ$C, rising by 1$^\circ$C, 45 seconds, 5 seconds.

\subsection{ChIP-seq}

\textbf{(1) ChIP:} ChIP was carried out according to \textbf{Section \ref{chipmethod}}. About $5 \times 10^7$ U2OS cells, 4.5 $\mu$g antibodies, and 45 $\mu$L Dynabeads were used for each experiment. Briefly, the cells were washed in 30 mL ChIP Lysis Buffer I, 30 mL ChIP Lysis Buffer II at 4$^\circ$C, rotating for 5 minutes for each wash. Then the nuclei were lysed in 3 mL ChIP Lysis Buffer III, sonicated for 45 minutes. After addition of Triton, the lysates were split into 3 equal aliquots to perform ChIP individually. Therefore, each individual ChIP contained 1 mL lysate, 1.5 $\mu$g antibody, and 15 $\mu$L Dynabeads. After overnight incubation at 4$^\circ$C, each ChIP was washed, eluted, reverse crosslinked, and purified individually. After DNA precipitation, 20 $\mu$L 10 mM Tris, pH 8.5 were used to resuspend the DNA pellet from each ChIP. Then purified DNA from three individual ChIPs were pooled, yielding a total of 60 $\mu$L ChIPed DNA. 10 $\mu$L were taken to perform ChIP-qPCR to check enrichment at known target loci. The rest \textasciitilde 50 $\mu$L were used for the following library preparation steps.

\textbf{(2) Library preparation:} The library of FOXO3 ChIP-seq sample was constructed using the SOLiD 4 (Life Technology) library preparation protocol according to the manufacturer's instructions. Briefly, ChIPed DNA samples were quantified by Qubit (Life Technology). 1 - 10 ng DNA were sheared by Covaris\sus{TM} S2 System (Life Technology), and end-repaired by End Polishing Enzyme 1 and 2. After purification, DNA was ligated with P1 and P2 adaptors. The ligated and purified DNA was run on a SOLiD\sus{TM} Library Size Selection gel. The correctly ligated DNA (size 200 - 230 bp) was collected and purified from the gel. The elutes from the gel underwent nick translation and amplification to prepare libraries. Then the libraries were clonally amplified onto P1-conjugated beads in emulsion PCR reactions with the following temperature profile: 95$^\circ$C 5 minutes, [93$^\circ$C 15 seconds, 62$^\circ$C 30 seconds, 72$^\circ$C 75 seconds] $\times$ 40 cycles, 72$^\circ$C 7 minutes, 4$^\circ$C hold. Sequencing was done by ABI SOLiD 4 Analyser. The library of FOXM1 ChIP-seq sample was done using Illumina/Solexa Genomic DNA protocol. Briefly, the ChIP sample or 5 - 50 ng Input DNA were end-repaired and ligated with the Illumina Genomic Adaptor Oligo. Then the Adaptor-modified DNA was amplified using Genomic PCR primer 1.1/2.1 (Illumina) with the following temperature profile: 98$^\circ$C 30 seconds, [98$^\circ$C 10 seconds, 65$^\circ$C 30 seconds, 72$^\circ$C 30 seconds] $\times$ 18 cycles, 72$^\circ$C 5 minutes, 4$^\circ$C hold. The amplified library was run on a 2\% agarose TAE gel, and the DNA of the size about 200 - 300 bp was excised and purified. Sequencing was done by the Illumina Genome Analyzer II.

\subsection{Bioinformatics analyses}

\textbf{(1) Mapping of tags to reference genome:} For FOXM1 ChIP-seq experiments, base calling was performed using the bundled Illumina image extraction pipeline, and the 36-bp tag information was stored in the fastq file. Tags were aligned against NCBI Build 36.3/hg18 of the human genome using MAQ (\url{http://maq.sourceforge.net/}) allowing up to 2 mismatches. Only reads that were uniquely mapped to the genome were preserved. For FOXO3 ChIP-seq experiments, base calling was performed by the bundled SOLiD image extraction pipeline, and the 50-bp tag information was stored in the csfasta file. Tags were aligned against NCBI Build 36.3/hg18 of the human genome using BFAST (\url{http://bfast.sourceforge.net/}) (\cite{homer2009bfast:}). Tags that were uniquely mapped to the genome were preserved. For those tags which could be mapped to multiple regions of the genome, only the tags with the best score were preserved.

\textbf{(2) Identification of TF binding sites (peak calling):} Reads that mapped to the mitochondrial genome were removed. MACS (version 1.4.1) and HOMER (version 3.10) were used to call peaks using their default parameters respectively. For getting high confidence FOXK2 and FOXO3 peaks, peak calling results from MACS and HOMER were overlapped, and the peak coordinates from MACS were used for downstream analysis.

\textbf{(3) \textit{De novo} motif discovery:} Sequences of 200-bp regions centred on the summits of the peaks (generated by the peak caller MACS which identifies the position of highest tag pileup for each peak) were put into HOMER or MEME. Motif searches were carried out using their own default settings. In HOMER, background sequences were randomly selected from the genome with the same length and distribution of GC-content as the target sequences; in MEME, background sequences were created by shuffling the target sequences.

\textbf{(4) Known motif enrichment analysis:} HOMER was used to screen the enrichment of a known motif in the target sequences relative to the background sequences which were selected as described in \textbf{\textit{De novo} motif discovery}. The significance of the enrichment of a known motif was represented by the \textit{P}-value which was calculated by a hypergeometric distribution. The motif enrichment heatmap matrix was created using the $-log_{10}P$-values, and the matrix was visualised by Multi-Experiment viewer (\cite{saeed2003tm4:}).

\textbf{(5) Overlapping of binding sites:} The BED file containing the coordinates of the peaks was uploaded to the Galaxy genome browser. The overlap was analysed using the function `Join genomic intervals' with 1 bp minimum overlap.

\textbf{(6) Tag counts heatmap and cluster:} Peaks were centred on their summit, and a region of 5 kb upstream and 5 kb downstream relative to the summit was used. The 10-kb regions was divided into 200 bins (50-bp per bin), and the number of overlapping reads were counted within each bin and normalised to 10 million total reads. The data matrix was generated by HOMER or seqMiner, and the heatmap was visualised by JavaTreeView (\url{http://jtreeview.sourceforge.net/}). For clustering analysis, k-means linear method in the seqMiner was used to identify similar binding profiles.

\textbf{(7) Gene ontology:} The summit positions of the peaks were put into GREAT web tool, and gene ontology analyses were performed using the default parameters of basal plus extension method of GREAT.

\textbf{(8) Lift-over:} For comparison among the ChIP-seq data sets of FOXM1, LIN9 and B- MYB, FOXM1 ChIP-seq data was converted to hg19 assembly using the UCSC LiftOver tool (\url{http://hgdownload.cse.ucsc.edu/admin/exe/}).

\textbf{(9) Gene expression analyses:} The CEL files of the MuvB microarray data in HeLa cells were downloaded from GEO (accession number GSE27031). Z scores of each of the FOXM1 bound genes assigned by GREAT were calculated using MATLAB. The scores were used to generate the heatmap matrix, and the heatmap dendrogram was visualized using Multi-Experimental Viewer (\cite{saeed2003tm4:}) to present differential expression patterns throughout the cell cycle.

\subsection{\textit{In vitro} DNA pulldown assay}

DNA for the \textit{in vitro} pulldown assay was amplified by PCR using the plasmids pGL4.1 (Promega), pAS3017, pAS3018, pAS3019, pAS3020, pAS3036 and pAS3050 using different biotin labelled primers listed in \textbf{Table \ref{table:oligodnapd}}. PCR reactions were carried out according to \textbf{Section \ref{section:pcrclone}}.

PCR products were purified using a Qiagen PCR purification kit, and eluted into 50 $\mu$L 10 mM Tris, pH 8.5 (Elution buffer in the kit) at the final step. The 50 $\mu$L eluted DNA was mixed with equal volume of 2$\times$ Biotin binding buffer, and 50 $\mu$L M280 Streptavidin Dynabeads (Invitrogen) were added to the mixture. The binding reaction was left at room temperature for 20 minutes with occasional agitation. After binding the DNA to the beads, the DNA-beads complex was washed with 1$\times$ Biotin binding buffer for twice, and HKMG buffer for three times.

One 10-cm dish of proliferating U2OS cells were lysed in 500 $\mu$L HKMG buffer by sonication using Bioruptor UCD200 (Diagenode) for 5 minutes, high power. The cell debris was removed by centrifugation at 13,200 rpm, 4$^\circ$C, for 10 minutes. The cell lysates were incubated with the washed DNA-beads complexes at 4$^\circ$C on a rotating platform overnight. Then the beads were washed four times with HKMG buffer, and elution was performed by boiling the beads in 1$\times$ SDS loading buffer. The elutes were run on a 10\% acrylamide gel, and the protein was analysed by western blot.

\subsection{RNA isolation and reverse transcriptase quantitative PCR}

Cells were cultured in 12-well plates. 48 hours after siRNA transfection, culture media were removed, and cells were rinsed with 1$\times$ PBS once, and RNA was purified using an RNeasy Kit (Qiagen) according to manufacturer's protocol. The RNA concentration was measured by a NanoDrop Spectrophotometer ND-1000 (Thermo Scientific) after incubation at 65$^\circ$C for 10 minutes, and 20 ng total RNA were used for reverse transcriptase quantitative PCR (RT-qPCR). The reaction contained:

\begin{tabular}{>{\raggedleft\arraybackslash}m{1cm}>{\raggedright\arraybackslash}m{1.5cm}>{\raggedright\arraybackslash}m{8cm}}
    5    & $\mu$L    & SYBR RT Master Mix (Qiagen)\\
    0.1  & $\mu$L    & RT mix\\
    0.6  & $\mu$L    & 50 $\mu$M primer mix\\
    2.2  & $\mu$L    & H\sub{2}O\\
    2    & $\mu$L    & RNA template (20 ng in total)\\
\end{tabular}

The reaction was incubated at the following temperature profile: 50$^\circ$C 30 minutes, 95$^\circ$C 15 minutes, [95$^\circ$C 20 seconds; 57$^\circ$C 30 seconds; 72$^\circ$C 30 seconds] $\times$ 40 cycles; melting curve: 72 - 99$^\circ$C, rising by 1$^\circ$C, 90 seconds, 5 seconds.

\subsection{Primer design}

For ChIP primers, DNA sequences from non-repeat regions of the genome (NCBI/hg18) were extracted using the UCSC genome browser, and primers were designed using the Primer3 online application (\url{http://biotools.umassmed.edu/bioapps/primer3_www.cgi}) using Human Mispriming library, with primer size at 18 - 27 bp, and product size at 100 - 250 bp. The output primer pairs were tested using UCSC \textit{in silico} PCR, and only primers given one single product were chosen. For RT-qPCR, genomic sequence and mRNA sequence of a gene were retrieved from the Refseq database, and put into PerlPrimer (\url{http://perlprimer.sourceforge.net/}).  The primer size was set as 20 - 24 bp, and the product size was set at 100 - 250 bp. Only primers that were annealed to exon-exon junction were chosen.

Then primers were tested using regular PCR, and the products were run on an agarose gel. Only primers that give one single band at the expected size were preserved.

\subsection{Statistical analysis}

The standard deviation ($s$) was calculated using the formula:

\[
s = \sqrt{\frac{1}{N-1}\sum_{i=1}^N(x_i - \bar{x})^2}
\]

Data were shown as means $\pm s$ of at least two independent experiments. $P$-values based on independent one/two-tailed Student's $t$-tests were calculated using the Microsoft Excel function TTEST.

Fisher's exact tests (small sample size) and chi-square (large sample size) tests were carried out using an online tool (\url{http://graphpad.com/quickcalcs/contingency1.cfm}), and two-tailed $P$-values were reported.

Other statistical tests including hypergeometric tests, binomial tests and Poisson statistics were carried out by the software described in \textbf{Chapter \ref{ch:results}}.
